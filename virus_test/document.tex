\documentclass[]{article}
\usepackage{amsmath}
\usepackage{amssymb}
\usepackage{amsfonts}
\usepackage{mathtools}

\newcommand{\ud}{\mathrm{d}}

\begin{document}
\section{The problem}
A large number of samples are waiting to be tested. Each sample has an independent probability $p$ of being positive, where $p \ll 1$. To speed up testing, multiple sample can be mixed into one for test, and the test result will be positive if and only if at least one sample before mixing is positive. Assuming the test is $100\%$ accurate no matter how diluted the sample is, what is the best strategy to minimize the expectation of total number of tests to perform?

\section{The background}

During COVID-19 pandemic, Wuhan performed testing on the entire population of the city. Given the low positive rate expected at that time, they performed "group testing" to speed up the process: each 10 samples are grouped together and the mixture of them is tested first. If the mixture is positive, individual sample in the group is then tested.

\section{The group-once strategy}
\subsection{Derivation}
We can try to derive the optimal strategy following Wuhan style: we group $x$ samples and perform group testing. The question is, for a given positive rate $p$, what the best $x$ is which gives the lowest expectation on test count.

For one group, there are three possible outcomes:
\begin{itemize}
	\item all samples are negative, in which case we only need to test once for the mixture. The probability of this case is $(1-p)^x$.
	\item the last one is positive, and all others are negative. In this case we first performs a test on the mixture, then perform $x - 1$ tests for individuals before the last one. Once all these individual tests come back negative, we know the last one must be positive without needing to test it. The total test count is $x$. The probability of this case is $p(1-p)^{x - 1}$.
	\item in all other cases, we need to test all individual samples after the mixture test, which means $x + 1$ tests to perform, and the probability is $1 -(1-p)^x -  p(1-p)^{x - 1} = 1 - (1-p)^{x - 1}$.
\end{itemize}

So the expectation of test count for this group is
\begin{align*}
& (1-p)^x + xp(1-p)^{x - 1} + (x + 1)( 1 - (1-p)^{x - 1}) \\
=\,& x(1-(1-p)^x)  + 1 - p(1-p)^{x - 1}\,,
\end{align*}
and the expectation of test per sample is
\begin{align*}
f(x) = 1-(1-p)^x + \frac{1}{x}(1 - p(1-p)^{x - 1})\,.
\end{align*}
\subsection{Continuity}
We want to find the minimal $f(x)$ for positive integer $x$. It is hard to work directly on the discrete problem, so we first consider the continuous function $f(x)$ over $x \in \mathbb{R}^+$. The derivative of the function is
\begin{align*}
f'(x) &= -(1-p)^x \ln(1-p) - \frac{1}{x}p(1-p)^{x - 1}\ln(1-p) - \frac{1}{x^2}(1 - p(1-p)^{x - 1})\\
&= \frac{1}{x^2}( (1-p)^{x-1}(-(1-p)\ln(1-p)x^2 - p\ln(1-p)x + p) - 1 )\,.
\end{align*}
We want to analyze where $f'(x) \lessgtr 0$, which is
\[
-\ln(1-p)x^2 - \frac{p}{1-p}\ln(1-p)x + \frac{p}{1-p} \lessgtr (1-p)^{-x}\,.
\]
Let $-x\ln(1-p) = y$, then it is 
\[
g(y) = -\frac{1}{\ln(1-p)}y^2 +\frac{p}{1-p}y +\frac{p}{1-p} \lessgtr e^y\,.
\]
$g(y)$ is a parabola open upward, which can intersect with $e^y$ at most three times. To find the range for those intersections, we can test that
\begin{itemize}
	\item $g(-\infty) > 0 = e^{-\infty}$,
	\item $g(0) = p/(1 - p) < 1 = e^0$ when $p < 1/2$, 
	\item $g(1) = -1/\ln(1-p) + 2p/(1-p)> e$, (??)
	\item $g(\infty) = O(y^2) < O(e^y) = e^{\infty}$.
\end{itemize}
So for a small positive $p$, there are three intersections at $y_0\in(-\infty, 0)$, $y_1\in(0, 1)$, and $y_2\in(1, \infty)$, respectively, and we also learn the sign of $g(y)$ between them. We can then know the existence of the respective extremum $x_1$ and $x_2$ (discarding the negative  $y_0$). We can also find the function limit on the boundary
\begin{align*}
\lim_{x\to0^+} f(x) &= +\infty\\
\lim_{x\to+\infty} f(x) &= 1\,.
\end{align*}
The overall shape of $f(x)$ for $p < 1/2$ is
\begin{table}[!h]
	\centering
	\caption{Monotonicity and local extremum of $f(x)$}
	\begin{tabular}{c|ccccccc}
		\hline
		$x$&$0$&$(0, x_1)$&$x_1$&$(x_1, x_2)$&$x_2$&$(x_2, +\infty)$&$+\infty$\\
		\hline
		$f'(x)$&&$-$&$0$&$+$&$0$&$-$&\\
		\hline
		$f(x)$&$+\infty$&$\searrow$&min&$\nearrow$&max&$\searrow$&$1$\\
		\hline
	\end{tabular}
	\label{tab:g}
\end{table}

In the following, we will only discuss the case where $x_1 > 1$. This corresponds to $f'(1) < 0$, which is
\begin{align*}
-(1-p)\ln(1-p) - p\ln(1-p) + p - 1  &< 0\\
 \ln(1-p) + 1-p &> 0\\
 (1 - p)e^{1-p} &> 1\\
 1 - p &>W(1)\\
 p &<1-W(1)\approx 0.43
\end{align*}
Note that $f(1) = 1$, so $x_1 > 1$ gives $f(x_1) < 1$. There is no other region with value less than 1, so $x_1$ is the global minimal point.

\subsection{Confinement}
We will prove the following statement that makes a good approximation to $x_1$:\\
For a small enough $p$ (we use $0 < p < 0.1$ for this prove), we always have \[\frac{1}{\sqrt{p}} < x_1 < \frac{1}{\sqrt{p}} + \frac{1}{2}\,.\]
Given the monotonicity analysis above, this is equivalent to\\ \[f'\left(\frac{1}{\sqrt{p}}\right) < 0 < f'\left(\frac{1}{\sqrt{p}} + \frac{1}{2}\right)\,.\]

We will formulate our proof in the reverse order: antecedent will be placed at the bottom.
\begin{align*}
&f'\left(\frac{1}{\sqrt{p}}\right) < 0\\
\Longleftarrow\quad&-\frac{1}{p}(1-p)\ln(1-p) - \sqrt{p}\ln(1-p) + p < (1-p)^{1-\frac{1}{\sqrt{p}}}\\
\Longleftarrow\quad&-\ln(1-p)\left(\frac{1}{p} - 1 + \sqrt{p}\right)< (1-p)^{1-\frac{1}{\sqrt{p}}} - p\\
\Longleftarrow\quad&-\ln(1-p)\left(\frac{1}{p} - 1 + \sqrt{p}\right)< 1\quad\wedge\quad 1 < (1-p)^{1-\frac{1}{\sqrt{p}}} - p\ ,
\end{align*}

\begin{align*}
&-\ln(1-p)\left(\frac{1}{p} - 1 + \sqrt{p}\right)< 1\\
\Longleftarrow\quad& -\ln(1-p) < \frac{p}{1 - p + p^{3/2}}\\
\Longleftarrow\quad& -\ln(1-p) < \frac{3}{5}p^2+p\quad\wedge\quad\frac{3}{5}p^2+p < \frac{p}{1 - p + p^{3/2}}\ ,
\end{align*}

\begin{align*}
& -\ln(1-p) < \frac{3}{5}p^2+p\\
\Longleftarrow\quad&h(p) = -\ln(1-p) - \frac{3}{5}p^2 - p < 0\\
\Longleftarrow\quad&h(0) = 0\ \mbox{(trivial)} \quad\wedge\quad h'(p) < 0\ \mbox{for}\ 0 < p < 0.1\\
\Longleftarrow\quad&\frac{1}{1-p} -  \frac{6}{5}p - 1 < 0\\
\Longleftarrow\quad&-\frac{1}{5}p + \frac{6}{5}p^2  < 0 \\
\Longleftarrow\quad& 0 < p < \frac{1}{6} \ \mbox{(given)}\ ,
\end{align*}

\begin{align*}
&\frac{3}{5}p^2+p < \frac{p}{1 - p + p^{3/2}}\\
\Longleftarrow\quad&\left(\frac{3}{5}p+1\right)(1 - p + p^{3/2}) < 1\\
\Longleftarrow\quad&  3p^{3/2}- 3p  + 5p^{1/2} -2 < 0\\
\Longleftarrow\quad& j(u) = 3u^3 -3u^2+5u-2 < 0\ (u = \sqrt{p})\\
\Longleftarrow\quad& j(0.1) < 0\ \mbox{(trivial)} \quad\wedge\quad j'(u) > 0 \ \mbox{for}\ 0 < p < 0.1\\
\Longleftarrow\quad& 9u^2 - 6u + 5 > 0 \ \mbox{(true for all $u$)}\ ,
\end{align*}

\begin{align*}
 &1 < (1-p)^{1-\frac{1}{\sqrt{p}}} - p\\
\Longleftarrow\quad& \ln(1 + p) < \left(1-\frac{1}{\sqrt{p}}\right)\ln(1-p)\\
\Longleftarrow\quad& \ln(1 + p) <p\ \mbox{(trivial)}  \quad\wedge\quad p< \left(1-\frac{1}{\sqrt{p}}\right)\ln(1-p)\\
\Longleftarrow\quad&\frac{p^{3/2}}{1-\sqrt{p}}< -\ln(1-p)\\
\Longleftarrow\quad&\frac{p^{3/2}}{1-\sqrt{p}}<p \quad\wedge\quad p< -\ln(1-p)\ \mbox{(trivial)}\\
\Longleftarrow\quad&p^{3/2}<p - p^{3/2}\\
\Longleftarrow\quad&2p^{3/2}<p\\
\Longleftarrow\quad&p<\frac{1}{4} \ \mbox{(given)}\ ;
\end{align*}

\begin{align*}
&f'\left(\frac{1}{\sqrt{p}}+\frac{1}{2}\right) > 0\\
\Longleftarrow\quad& (1-p)^{\frac{1}{\sqrt{p}}-\frac{1}{2}}\left(-(1-p)\ln(1-p)\left(\frac{1}{\sqrt{p}}+\frac{1}{2}\right)^2 - p\ln(1-p)\left(\frac{1}{\sqrt{p}}+\frac{1}{2}\right) + p\right) > 1\\
\Longleftarrow\quad&-\ln(1-p)\left(\frac{p}{4}+\frac{1}{\sqrt{p}}+\frac{1}{p}-\frac{3}{4}\right) > (1-p)^{\frac{1}{2}-\frac{1}{\sqrt{p}}} - p\\
\Longleftarrow\quad&-\ln(1-p) > p + \frac{p^2}{2}  \quad\wedge\quad\left(p + \frac{p^2}{2}\right) \left(\frac{p}{4}+\frac{1}{\sqrt{p}}+\frac{1}{p}-\frac{3}{4}\right) > \frac{1 + \sqrt{p}}{\sqrt{1-p}} - p \\& \quad\wedge\quad \frac{1 + \sqrt{p}}{\sqrt{1-p}} >  (1-p)^{\frac{1}{2}-\frac{1}{\sqrt{p}}}\ ,
\end{align*}

\begin{align*}
&-\ln(1-p) > p + \frac{p^2}{2}\\
\Longleftarrow\quad& k(p) = -\ln(1-p) - p - \frac{p^2}{2} > 0\\
\Longleftarrow\quad& k(0) = 0 \ \mbox{(trivial)} \quad\wedge\quad k'(p) > 0 \ \mbox{for}\ 0 < p < 0.1\\
 \Longleftarrow\quad& \frac{1}{1 - p} - 1 - p > 0\\
  \Longleftarrow\quad& 1 > 1 - p^2  \ \mbox{(trivial)} \ ,
\end{align*}

\begin{align*}
&\left(p + \frac{p^2}{2}\right) \left(\frac{p}{4}+\frac{1}{\sqrt{p}}+\frac{1}{p}-\frac{3}{4}\right) > \frac{1 + \sqrt{p}}{\sqrt{1-p}} - p\\
\Longleftarrow\quad&\left(u^2 + \frac{u^4}{2}\right) \left(\frac{u^2}{4}+\frac{1}{u}+\frac{1}{u^2}-\frac{3}{4}\right) > \frac{1 + u}{\sqrt{1-u^2}} - u^2 = \sqrt{\frac{1 + u}{1-u}} - u^2\ (u = \sqrt{p})\\
\Longleftarrow\quad&\frac{u^6}{8}-\frac{u^4}{8}+\frac{u^3}{2}+\frac{3u^2}{4}+u+1 >\sqrt{\frac{1 + u}{1-u}}\\
\Longleftarrow\quad&   \frac{u^6}{8}-\frac{u^4}{8}+\frac{u^3}{2}+\frac{3u^2}{4}+u+1  > 0   \ \mbox{(trivial)} \\&\quad\wedge\quad  \left(\frac{u^6}{8}-\frac{u^4}{8}+\frac{u^3}{2}+\frac{3u^2}{4}+u+1 \right)^2(1-u)-(1+u) > 0\\
\Longleftarrow\quad&-\frac{u^{13}}{64}+\frac{u^{12}}{64}+\frac{u^{11}}{32}-\frac{5u^{10}}{32}-\frac{5u^9}{64}+\frac{5u^8}{64}-\frac{3u^7}{16}-\frac{3u^6}{16}-\frac{13u^5}{16} - \frac{19u^4}{16}+\frac{u^2}{2} > 0\\
\Longleftarrow\quad&-\frac{5u^{10}}{32}-\frac{5u^9}{64}-\frac{3u^7}{16}-\frac{3u^6}{16}-\frac{13u^5}{16} - \frac{19u^4}{16} > -\frac{u^2}{2}\\
\Longleftarrow\quad&l(u) = \frac{5u^{8}}{32}+\frac{5u^7}{64}+\frac{3u^5}{16}+\frac{3u^4}{16}+\frac{13u^3}{16} + \frac{19u^2}{16} < \frac{1}{2}\\
\Longleftarrow\quad&l'(u) > 0\ \mbox{(trivial)} \ \quad\wedge\quad l(\sqrt{0.1}) < \frac{1}{2}\ \mbox{(trivial)}
\end{align*}

\begin{align*}
&\frac{1 + \sqrt{p}}{\sqrt{1-p}} >  (1-p)^{\frac{1}{2}-\frac{1}{\sqrt{p}}}\\
\Longleftarrow\quad& 1 + \sqrt{p} >  (1-p)^{1-\frac{1}{\sqrt{p}}}\\
\Longleftarrow\quad& \ln(1+\sqrt{p}) > \left(1-\frac{1}{\sqrt{p}}\right)\ln(1-p) =\left(1-\frac{1}{\sqrt{p}}\right)(\ln(1+\sqrt{p}) + \ln(1 - \sqrt{p})) \\
\Longleftarrow\quad& \frac{1}{\sqrt{p}}\ln(1+\sqrt{p}) + (\frac{1}{\sqrt{p}}-1)\ln(1-\sqrt{p}) > 0\\
\Longleftarrow\quad& \ln(1+\sqrt{p}) + (1-\sqrt{p})\ln(1-\sqrt{p}) > 0\\
\Longleftarrow\quad& m(u) = \ln(1+u) + (1-u)\ln(1-u) > 0\ (u = \sqrt{p})\\
\Longleftarrow\quad& m(0) = 0 \ \mbox{(trivial)} \ \quad\wedge\quad m'(u) > 0 \ \mbox{for}\ 0 < p < 0.1 \\
\Longleftarrow\quad& \frac{1}{1+u} - \ln(1-u) - 1 > 0\\
\Longleftarrow\quad&  \frac{u}{1+u} < - \ln(1-u)\\
\Longleftarrow\quad&  \frac{u}{1+u} <u\ \mbox{(trivial)} \  \quad\wedge\quad u < - \ln(1-u)\ \mbox{(trivial)} \ 
\end{align*}
\end{document}