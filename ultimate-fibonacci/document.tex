\documentclass[]{article}
\usepackage{amsmath}
\DeclareMathOperator\Arg{Arg}
\usepackage{amsthm}
\usepackage{amssymb}
\usepackage{mathtools}
\usepackage
[
left=2cm,
right=2cm,
top=2cm,
bottom=2cm,
]
{geometry}

\setlength\parskip{1em}

\begin{document}

\newtheorem{problem}{Problem}
\newtheorem{lemma}{Property}

\section{The problem}

Function $f: \mathbb{R}\to\mathbb{C}$ satisfies the recurrence relation
\[
\lambda_1 f(t+t_1) + \lambda_2 f(t+t_2) + \dots \lambda_n f(t+t_n) = 0
\]
where
\[
0 = t_1 < t_2 <\dots <t_n = T
\]
Given the function value $f(t)$ for $t\in[S, S+T]$, find the general formula for $f(t)$

\section{Solution}

Consider the following solution basis that satisfies the recurrence relation
\[
p_j(t) = e^{x_j t}
\]
which means that $x_j$ are the solutions for the following equation
\[
\sum_{i = 1}^{n} \lambda_i e^{x_j t_i} = 0
\]
For now let's assume there is no repeated solutions. We want to express $f(t)$ in terms of the linear combination
\[
f(t) = \sum_j c_j e^{x_j t}
\]
To find the coefficients, we define the following dual basis
\[
q_j(t) = -\frac{\sum_{i=1}^{t_i<t-S} \lambda_i e^{x_j(t_i-t)} }{\sum_{i=1}^{n} \lambda_i t_i e^{x_j t_i}}
\]
These dual basis has the following nice property with the original basis
\begin{align*}
\int_{S}^{S+T} p_j(t)q_k(t)\,dt &= \int_{S}^{S+T} -e^{x_j t} \frac{\sum_{i=1}^{t_i<t-S} \lambda_i e^{x_k( t_i-t)} }{\sum_{i=1}^{n} \lambda_i t_i e^{x_k t_i}}\,dt \\
&= -\frac{\sum_{i=1}^{n} \int_{S+t_i}^{S+T}  \lambda_i e^{x_k t_i}e^{(x_j - x_k) t}\,dt }{\sum_{i=1}^{n} \lambda_i t_i e^{x_k t_i}}\\
&= \begin{dcases*}
	 -\frac{\sum_{i=1}^{n}  \lambda_i e^{x_k t_i}  (T-t_i) }{\sum_{i=1}^{n} \lambda_i t_i e^{x_k t_i}} & $(x_j =x_k)$\\
	 -\frac{\sum_{i=1}^{n} \lambda_i \left(e^{(x_j-x_k)(S+T)}e^{x_kt_i} - e^{(x_j-x_k)S}e^{x_jt_i}\right)}{(x_j - x_k)\sum_{i=1}^{n} \lambda_i t_i e^{x_k t_i}} & $(x_j \ne x_k)$
\end{dcases*}\\
&=\delta_{j}^k
\end{align*}
This means we can extract the coefficient using the dual basis
\begin{align*}
c_j &= \int_{S}^{S+T} f(t)q_j(t)\,dt \\
    &=   -\frac{\sum_{i=1}^{n} \int_{S+t_i}^{S+T} \lambda_i e^{x_j(t_i-t)}f(t)\,dt }{\sum_{i=1}^{n} \lambda_i t_i e^{x_j t_i}}
\end{align*}

\section{Periodic}
A periodic function is a special case of this problem where
\begin{align*}
	&n = 2 \\
	&\lambda_1 = 1,\quad\lambda_2 = -1\\
	&t_1 = 0,\quad t_2 = T
\end{align*}
for which we have the equation
\[
1 - e^{Tx} = 0
\]
which has solutions
\[
x_j = \frac{2\pi i j}{T}
\]
The solution bases and the dual basis are
\[
p_j(t) = e^{\frac{2\pi i jt}{T}}
\]
\[
q_j(t) =  -\frac{ e^{-x_jt} }{ -T e^{x_j T}} = \frac{1}{T}e^{-\frac{2\pi i jt}{T}}
\]
which is equivalent to finding the Fourier series of $f(t)$

\section{Distribution}
The original problem can be generalized to to the recurrence relation with distribution $\lambda(\theta)$ with a support within $[0, T]$
\[
\int_{0}^{T}\lambda(\theta)f(t + \theta)\, d\theta = 0
\]
The solution basis are
\[
p_j(t) = e^{x_jt}
\]
where $x_j$ are solutions to the equation
\[
\int_{0}^{T}\lambda(\theta)e^{x_j\theta}\,d\theta = 0
\]
And the dual basis are
\[
q_j(t) = -\frac{\int_0^{t-S} \lambda(\theta)e^{x_j(\theta-t)}\,d\theta}{\int_0^T\theta\lambda(\theta)e^{x_j\theta}\,d\theta}
\]
which has the property
\begin{align*}
	\int_S^{S+T} p_j(t)q_k(t)\,dt &= \int_S^{S+T} -e^{x_jt}\frac{\int_0^{t-S} \lambda(\theta)e^{x_k(\theta-t)}\,d\theta}{\int_0^T\theta\lambda(\theta)e^{x_k\theta}\,d\theta}\,dt \\
	&= -\frac{\int_{\theta=0}^{T}\int_{t=S+\theta}^{S+T} \lambda(\theta)e^{x_jt}e^{x_k(\theta-t)}\, dt d\theta}{\int_0^T\theta\lambda(\theta)e^{x_k\theta}\,d\theta } \\
	&=\begin{dcases*}
		-\frac{\int_{0}^{T} (T-\theta)\lambda(\theta)e^{x_k\theta}\,d\theta}{\int_0^T\theta\lambda(\theta)e^{x_k\theta}\,d\theta } & $(x_j = x_k)$\\
		-\frac{\int_{0}^{T}\lambda(\theta)\left(e^{(x_j-x_k)(S+T)}e^{x_k\theta} - e^{(x_j-x_k)S}e^{x_j\theta}\right)\,d\theta }{(x_j-x_k)\int_0^T\theta\lambda(\theta)e^{x_k\theta}\,d\theta }  & $(x_j \ne x_k)$
	\end{dcases*}\\
	&=\delta_j^k
\end{align*}
So we can extract the coefficients the same way as before.

\section{Discrete}
A discrete analog to the original problem is as follows:

Sequence $f: \mathbb{Z}\to\mathbb{C}$ satisfies the recurrence relation
\[
\lambda_{0} f(t) + \lambda_1 f(t+1) + \dots + \lambda_T f(t+T) = 0
\]
where $\lambda_{0} \ne 0$ and $\lambda_T \ne 0$. Given the sequence value $f(S), f(S+1),\dots,f(S+T-1)$, find the general formula for $f(t)$.

We can consider the solution basis that satisfies the recurrence relation
\[
p_j(t) = x_j^t
\]
which means $x_j$ are the solutions to the following algebraic equation
\[
\sum_{n = 0}^{T} \lambda_{n} x^n = 0 
\]
This equation has $T$ solutions. For now, let's assume all ${x_j}$ are distinct. We want to express $f(t)$ in terms of the linear combination
\[
f(t) = \sum_j c_j x_j^t
\]
To find the coefficients, we define the following dual basis
\[
q_j(t) = -\frac{\sum_{n=0}^{t-S}\lambda_n x_j^{n-t}}{\sum_{n=0}^{T}\lambda_n n x_j^{n}}
\]
which has the property
\begin{align*}
	\sum_{t=S}^{S+T-1} p_j(t) q_k(t) &= -\frac{\sum_{t=S-S}^{S+T-1} \sum_{n=0}^{t-S}\lambda_n x_k^{n-t} x_j^t }{\sum_{n=0}^{T}\lambda_n n x_k^{n}} \\
	&= -\frac{\sum_{n=0}^{T-1}\sum_{t=S+n}^{S+T-1} \lambda_n x_k^{n-t} x_j^t }{\sum_{n=0}^{T}\lambda_n n x_k^{n}} \\
	&= \begin{dcases*} 
		 -\frac{\sum_{n=0}^{T-1}(T-n) \lambda_n x_j^{n} }{\sum_{n=0}^{T}\lambda_n n x_k^{n}} & $(x_j = x_k)$\\
		-\frac{\sum_{n=0}^{T-1}\lambda_n \left((x_j/x_k)^{S+T}x_k^n - (x_j/x_k)^S x_j^n \right)  }{(x_j/x_k - 1)\sum_{n=0}^{T}\lambda_n n x_k^{n}} & $(x_j \ne x_k)$
	\end{dcases*}\\
	&=\delta_{j}^k
\end{align*}
This means we can extract the coefficient using the dual basis
\begin{align*}
	c_j &= 	\sum_{t=S}^{S+T-1} f(t) q_j(t) \\
	&= -\frac{\sum_{t=S}^{S+T-1}\sum_{n=0}^{t-S}\lambda_n f(t) x_j^{n-t}}{\sum_{n=0}^{T}\lambda_n n x_j^{n}}
\end{align*}

\section{Fibonacci sequence}

The Fibonacci sequence is a special case where
\begin{align*}
	&T = 2\\
&	\lambda_0 = 1,\quad\lambda_1 = 1,\quad\lambda_2 = -1\\
	&S = 0\\
&	f(0) = 0,\quad f(1) = 1\\
\end{align*}
We have the equation for $x_j$
\[
1 + x - x^2 = 0
\]
and we have solution
\[
x_{0,1} = \frac{1\pm\sqrt{5}}{2}
\]
The dual basis are
\begin{align*}
q_{0,1}(0) &= -\frac{1}{x_{0,1}^1 - 2x_{0,1}^2}\\
&=\frac{5\mp\sqrt{5}}{10}  \\ 
q_{0,1}(1) &= -\frac{x_{0,1}^{-1} + 1}{x_{0,1}^1 - 2x_{0,1}^2}\\
&=\pm\frac{\sqrt{5}}{5} 
\end{align*}
So the coefficients are
\begin{align*}
	c_{0,1} = f(0)q_{0,1}(0) + f(1)q_{0,1}(1) = \pm\frac{\sqrt{5}}{5} 
\end{align*}
and the formula for the Fibonacci sequence is
\[
f(t) = c_0x_0^t + c_1x_1^t  =\frac{\sqrt{5}}{5} \left(\frac{1+\sqrt{5}}{2}\right)^t - \frac{\sqrt{5}}{5} \left(\frac{1-\sqrt{5}}{2}\right)^t
\]



%%%%%%%%%%%%%%%%%%%%%%%%%%%%%%%%%%%%%%%%%%%%%%%%%%%%%%%%%%%%%%%%%%%%%%%


\end{document}