\documentclass[]{article}
\usepackage{amsmath}
\usepackage{amssymb}
\usepackage{amsfonts}
\usepackage{mathtools}
\usepackage{amsthm}
\usepackage
[
left=2cm,
right=2cm,
top=2cm,
bottom=2cm,
]
{geometry}

\newcommand{\ud}{\mathrm{d}}
\theoremstyle{definition}\newtheorem{theorem}{Theorem}
\theoremstyle{definition}\newtheorem{lemma}{Lemma}

\newenvironment{subproof}[1][\proofname]{%
  \renewcommand{\qedsymbol}{$\blacksquare$}%
  \begin{proof}[#1]%
}{%
  \end{proof}%
}


\title{Grid circle}

\begin{document}
\maketitle

Sequence $\{a_n\}_{n=0}^{\infty}$ satisfies
\[
a_0 = 1
\]
\[
\sum_{k=0}^{n}a_k a_{n-k} = 1\quad \forall n\ge 0
\]

\begin{theorem}
	\[
	a_n = \frac{(2n-1)!!}{(2n)!!}
	\]
\end{theorem}
\begin{proof}
\begin{align*}
\frac{1}{\sqrt{1-x}} &= \sum_{n=0}^{\infty}  \frac{(2n-1)!!}{(2n)!!} x^{n}\\
\frac{1}{1-x} &= \left(\sum_{n=0}^{\infty}  \frac{(2n-1)!!}{(2n)!!} x^n\right)^2\\
&=\sum_{n=0}^{\infty} \left(\sum_{k=0}^{n}b_k b_{n-k}\right)x^n\quad\mbox{where } b_n = \frac{(2n-1)!!}{(2n)!!}
\end{align*}
However, we also have
\[
\frac{1}{1-x} = \sum_{n=0}^{\infty} x^n
\]
So
\[
\sum_{k=0}^{n}b_k b_{n-k} = 1
\]
\[
a_k = b_k
\]
\end{proof}

Sequence $\{x_n\}_{n=1}^{\infty}$ satisfies

\[
x_n=\sum_{k=0}^{n-1}a_k
\]

\begin{theorem}
\[
x_n = \frac{(2n-1)!!}{(2n-2)!!}
\]
\end{theorem}
\begin{proof}
By induction.
\[
x_1 = a_0 = 1 = \frac{1!!}{0!!}
\]
\begin{align*}
x_{n+1} = x_n + a_n = \frac{(2n-1)!!}{(2n-2)!!} + \frac{(2n-1)!!}{(2n)!!} = \frac{(2n-1)!!(2n) + (2n-1)!!}{(2n)!!} =  \frac{(2n+1)!!}{(2n)!!}
\end{align*}
\end{proof}

\begin{theorem}
For sequence ${x_n}$, if
\[
\lim_{n\to\infty} x_{n+1} - x_n = A
\]
\[
\lim_{n\to\infty} (n+1)x_n - nx_{n+1} = B
\]
then
\[
\lim_{n\to\infty} x_n - (An+B) = 0
\]
\end{theorem}
\begin{proof}
We denote that
\[
A_n = x_{n+1} - x_n
\]
\[
B_n = (n+1)x_n - nx_{n+1}
\]
We can also see 
\[
A_n = \frac{x_n - B_n}{n}
\]
To prove the theorem, we need to prove a lemma first
\begin{lemma}
If there exists a (large) integer $M$ such that $x_M = Y$, and for $n\geq M$, $B_n \in B + (-\delta, \delta)$, then for $n\geq M$, we have
\[
A_n \in \frac{Y - B}{M} + \left(-\frac{\delta}{M}, \frac{\delta}{M}\right)
\]
\[
x_n \in B + \frac{n(Y - B)}{M} + \left[-\frac{\delta(n-M)}{M}, \frac{\delta(n-M)}{M}\right]
\]
\end{lemma}
\begin{subproof}
By induction. For the initial condition we have
\[
x_M = Y \in B + \frac{M(Y - B)}{M} + \left[-0, 0\right]
\]
\[
A_M = \frac{x_M - B_M}{M} = \frac{Y - B_M}{M} =  \frac{Y - B}{M} + \left(-\frac{\delta}{M}, \frac{\delta}{M}\right)
\]
Assuming the statements about $A_n$ and $x_n$ are true for $n = k$, then for $n = k + 1$, we have
\begin{align*}
x_{k+1} = x_{k} + A_{k} &\in \frac{Y - B}{M} + \left(-\frac{\delta}{M}, \frac{\delta}{M}\right) + B + \frac{k(Y - B)}{M} + \left[-\frac{\delta(k-M)}{M}, \frac{\delta(k-M)}{M}\right]\\
 &= B + \frac{(k + 1)(Y - B)}{M} + \left[-\frac{\delta((k+1)-M)}{M}, \frac{\delta((k+1)-M)}{M}\right]
\end{align*}
and
\begin{align*}
A_{k+1} = \frac{x_{k+1} - B_{k+1}}{k+1} &\in \frac{1}{k+1}\left(B + \frac{(k + 1)(Y - B)}{M} + \left[-\frac{\delta((k+1)-M)}{M}, \frac{\delta((k+1)-M)}{M}\right] - B - (-\delta, \delta)\right)\\
& = \frac{1}{k+1}\left(\frac{(k + 1)(Y - B)}{M} + \left(-\frac{\delta(k+1)}{M}, \frac{\delta(k+1)}{M}\right)  \right)\\
& = \frac{Y - B}{M} + \left(-\frac{\delta}{M}, \frac{\delta}{M}\right)
\end{align*}
\end{subproof}

With the condition of Lemma 1 and an additional condition that $\lim_{n\to\infty}A_n = A$, we can see that
\[
A \in \frac{Y - B}{M} + \left(-\frac{\delta}{M}, \frac{\delta}{M} \right) 
\]
and for $n \geq M$
\[
|A_n - A| < \frac{2\delta}{M}
\]

Because $\lim_{n\to\infty}B_n = B$, by the definition of limit, for any (small) positive number $\delta$, there is always a $M$ that satisfy the condition of Lemma 1, and that
\[
|A_n - A|  < \frac{2\delta}{M}
\]
???

\end{proof}

\begin{theorem}
\[
\lim_{n\to\infty}x_n^2 - \frac{4n-1}{\pi} = 0
\]
\end{theorem}

\section{Let's go higher dimensions}

For some positive integer parameter $p$, sequence $\{a_n^{(p)}\}_{n=0}^{\infty}$ satisfies
\begin{align*}
	a_0^{(p)} &= 1\\
	\sum_{0\le i_1,i_2,\dots,i_p\le n}^{i_1+i_2+\dots+i_p=n} a_{i_1}^{(p)}a_{i_2}^{(p)}\dots a_{i_p}^{(p)} &= 1 \quad \forall n \ge 0
\end{align*}

\begin{theorem}
	\[
	a_{n}^{(p)} = \frac{1}{np}\cdot\left(\frac{1}{p} + 1\right)\left(\frac{1}{2p} + 1\right)\dots\left(\frac{1}{(n -1)p} + 1\right) = \frac{\Gamma\left(n + \frac{1}{p}\right)}{\Gamma(n+1)\Gamma\left(\frac{1}{p}\right)}
	\]
\end{theorem}
\begin{proof}
	Consider function 
	\[
	f(x) = (1-x)^{-1/p}
	\]
	which has derivatives
	\[
	f^{(n)}(x) = \frac{1}{p}\left(\frac{1}{p} + 1 \right)\dots\left(\frac{1}{p} + n -1\right)(1-x)^{-1/p-n}
	\]
	so it has Taylor series at $x=0$
	\[
	f(x) = \sum_{n=0}^{\infty} \frac{1}{n!}\cdot\frac{1}{p}\left(\frac{1}{p} + 1 \right)\dots\left(\frac{1}{p} + n -1\right) x^n
	\]
	At the same time, we have
	\[
	(f(x))^p = \frac{1}{1-x} = \sum_{n=0}^{\infty}x^n
	\]
	By comparing coefficients, we can prove the theorem
\end{proof}

\begin{theorem}
	Define sequence $\{x_n^{(p)}\}_{n=1}^{\infty}$ as the partial sum
	\[
	x_n^{(p)} = \sum_{k=0}^{n-1} a_k
	\]
	Then
	\[
	x_n^{(p)} = \left(\frac{1}{p} + 1\right)\left(\frac{1}{2p} + 1\right)\dots\left(\frac{1}{(n-1)p} + 1\right) = \frac{\Gamma\left(n + \frac{1}{p}\right)}{\Gamma(n)\Gamma\left(\frac{1}{p} + 1\right)} \sim \mathcal{O}(n^{1/p})
	\]
\end{theorem}
\begin{proof}
	By induction. Omitted
\end{proof}
\end{document}